\documentclass{scrartcl}

\usepackage[utf8]{inputenc}
\usepackage[T1]{fontenc}

\usepackage{booktabs}

\usepackage[ngerman]{babel}

\title{Einfache Tabelle}


\begin{document}

\listoftables



\section{Erläuterungen}

\begin{itemize}
	\item Das Basis-Element ist die tabular-Umgebung. Sie ist kein Float, sondern wird stur da eingefügt wo sie definiert wurde. 
	\item Um die tabular-Umgebung nun Eigenschaften eines Float zu geben, wird sie in eine table-Umgebung gefasst. Jetzt kann sie eine caption haben und taucht im Tabellenverzeichnis auf.
	\item Sehr häufig lädt man das booktabs-Package dazu. Es ändert ein wenig die Abstände, so dass man auf auf senkrechte Linien verzichten kann, welche Inhalte trennen. Ferner definiert es \verb|\toprule|, \verb|\midrule| und \verb|\bottomrule|: vertikale Linien, welche die Tabelle begrenzen.
\end{itemize}

\enlargethispage{3cm}


\section{Syntax}

Die Tabelle \ref{tab:zehnjahre} befindet sich auf Seite \pageref{tab:zehnjahre}. Hier ihr Code:

\begin{verbatim}
\begin{table}
\centering
\begin{tabular}{rcl}
\toprule
Zahl & Name & Bemerkung\\
\midrule
1 & text & langer Text\\
2 & blub & text\\
\end{tabular}
\caption{Das Ergebnis von 10 Jahren Forschung}
\label{tab:zehnjahre}
\end{table}
\end{verbatim}

\noindent Erläuterung:

\begin{itemize}
	\item Die äußere table-Umgebung konvertiert das Ganze in ein Float.
	\item Der von uns definierte Schlüssel in \verb|\label| kann mit \verb|\ref{}| und \verb|\pageref{}| verwendet werden. Dabei ist der Wert \verb|tab:zenhnjahre| frei gewählt und könnte komplett anderes lauten. Allerdings macht es Sinn anzudeuten auf welche Art von Element sich das Label bezieht (deshalb \verb|tab:|) und welches Element innerhalb dieser Gruppe das ist. 
	\item Die zweite geschweifte Klammer in \verb|\begin{tabular}{rcl}| definiert die Spalten:
	\begin{itemize}
		\item Jede Spalte muss hier definiert werden.
		\item $r$ meint right / rechtsbündig
		\item $c$ meint center / zentriert
		\item $l$ meint left / linksbündig
	\end{itemize}
	\item Das \verb|&| Symbol trennt die einzelnen Spalten voneinander.
	\item Sie beenden die Tabellenzeile immer mit \verb|\\|.
	\item alle Befehle im Code mit $rule$ sind von booktabs definierte Trennlinien unterschiedlicher Dicke.
\end{itemize}

\section{Die Tabelle}

Da wir die Tabelle in eine float-Umgebung gesetzt haben, bestimmt \LaTeX\ wo sie genau angezeigt wird. In diesem Fall vor der Überschrift, obowhl sie erst danach definiert wurde.\\
\\
In der Praxis empfiehlt es sich umfangreiche Tabellen in eine eigene Datei zu schreiben und diese mit \verb|\input{}| einzubinden.

\begin{table}
\centering
\begin{tabular}{rcl}
\toprule
Zahl & Name & Bemerkung\\
\midrule
1 & text & langer Text\\
2 & blub & text\\
\end{tabular}

\caption{10 Jahre Forschung}
\label{tab:zehnjahre}
\end{table}



\end{document}














