\documentclass{article}

% Interpretiert die Folge von 0 und 1 als utf8 Encoding
% und erlaubt so Umlaute, Accents, Sonderzeichen, ...:
\usepackage[utf8]{inputenc}

\usepackage[T1]{fontenc}

% Abmessungen des Papiers, Ränder et cetera mit geometry.
% Hier erst mal nur Festlegung auf DIN A4 statt US letter Format.
\usepackage[a4paper]{geometry}


% Beide Packete für Mathe
\usepackage{amsmath}
\usepackage{amssymb}

% blindtext erzeugt Texte, die nur dazu dienen Platz im Dokument zu füllem.
% So sehen wir im Kurs wie sich Umbrüche verhalten et cetera.
\usepackage{blindtext}

\usepackage[english,ngerman]{babel}

\usepackage{graphicx}

\usepackage{float}

% Folgende Angaben werden nur in den Text eingefügt,
% wenn dort ein \maketitle zu finden ist!
\title{\LaTeX-Beispieldokument}
\author{\foreignlanguage{ngerman}{Rüdiger Voigt, M.A.}}
\date{\today}

\usepackage{hyperref}
\hypersetup{
hidelinks = true
}

\begin{document}

\maketitle

\begin{abstract}
\blindtext
\end{abstract}

% Inhaltsverzeichnis
\tableofcontents

% Abbildungsverzeichnis
\listoffigures

% Erzwinge Seitenumbruch
\newpage

% Mit \input können wir tex-Dateien über ihren Dateinamen 
% einbinden. Auf die Endung .tex kann dabei verzichtet werden.
 \chapter{Einleitung}\label{chapterIntroduction}

\section{Querverweise}

Wenn Elemente ein Label haben, können Sie mit \verb|\ref| auf deren Nummerierung und mit \verb|\pageref| auf die Seite verweisen. Also zum Beispiel: Siehe Kapitel \ref{chapterIntroduction} auf Seite \pageref{chapterIntroduction}.\\
\ \\

\section{Sprache}

Um in anderen Sprachen als Englisch zu arbeiten, benötigen Sie das babel-package. Dann können Sie mit \verb|\foreignlanguage{}{}| mitten im Satz die Sprache wechseln:  \foreignlanguage{english}{\LaTeX\ is great!}

\section{Tabellen}

Die Float-Umgebung für Tabellen heißt table. Die eigentliche Tabelle ist eine tabular-Umgebung.

\begin{table}
\begin{center}
\begin{tabular}{rlc} % definiert die Spalten und deren Ausrichtung
% r = rechtsbündig
% l = linksbündig
% c = zentriert
\toprule % kommt aus booktabs
Variable & Wert & Beispiel\\
\midrule % braucht booktabs
a & 12345 6 & 3 \\
bbbbbbbbbbbbb & sedrftzuiopü\\
c & ertzuiop\\
\bottomrule
\end{tabular}
\label{tabMeine}
\caption{meine erste Tabelle}
\end{center}
\end{table}


\section{Neues aus der Wissenschaft}

\subsection{Subsection mit Nummerierung im Inhaltsverzeichnis}

\subsection*{Subsection, aber nicht im Inhaltsverzeichnis}
% Das Sternchen bewirkt, dass diese Subsection nicht im Inhaltsverzeichnis auftaucht.
% Sie wird auch nicht nummeriert, denn sonst gäbe es einen Sprung in der Nummerierung im 
% Inhaltsverzeichnis.
% Funktioniert so auch mit Kapiteln etc..

\subsection{Zeilenumbrüche}

Ein einfache Leerzeile

bewirkt, dass die erste Zeile des Folgeabsatzes eingerückt wird. Die zweite Zeile jedoch geht mit normalem linken Rand weiter und jede Folgende ebenso.

Das können Sie mittels Umbrüchen umgehen\footnote{Auch wenn das keine besonders schöbne Methode ist!}. \\
\ \\ % Achtung: hier muss ein erzwungenes Leerzeichen (Backslash + Leerzeichen) rein, sonst klappt der Umbruch nicht!
Das ist ein neuer Satz. 



\section{Text}

Eine Fußnote fügen Sie einfach mit dem Befehl \verb|\footnote{}| ein\footnote{So zum Beispiel}.\\
\\
Sie können mit \verb|\emph{}| \emph{Text hervorheben}.\\
\\
\textbf{Fettdruck} ist auch einfach.


\section{Floats}

\begin{figure}[htbp]
\centering
\includegraphics[width=0.5\textwidth]{img/2019-Dala-Mauer}
\label{img:cat}
\caption[Katze]{My cat in 2018}
\end{figure}

\section{Mathe}

Wenn Sie innerhalb des Fließtext mathematische Formeln einbinden möchten, umschließen Sie diese mit \$-Zeichen. Dann wird die Formel sehr kompakt dargestellt, wie zum Beispiel: $\sum_{i=1}^7$. Das macht nur Sinn für wenige komplexe Erläuterungen.
\\
Alternativ können Sie den abgesetzten Modus verwenden:

\[\sum_{i=1}^7\]


\noindent Die equation-Umgebung entspricht weitgehend dem abgesetzten Modus, aber fügt eine Nummerierung hinzu, die automatisch hochgezählt wird.

\begin{equation}
\sum = \pi \delta \$			
\end{equation}

\section{Abschnitt}

\Blindtext

\subsection{Referenzen}

Mit dem \verb|\ref{}| Befehl, können Sie ein vorher erstelltes Label referenzieren. Beispiel: Abbildung \ref{img:cat} auf Seite \pageref{img:cat}

\end{document}



















