\section{Mathematik}\label{sec:mathe}

Mittels \verb|S formel S| können Sie Formeln in den Fließtext einfügen. Dabei wird eine Formel wie 
$\sum_{i = 0}^{12} = \frac{\sqrt[3]{x^2}}{\pi}$ gestaucht.\\
\ \\
Um eine Formel im abgesetzten Modus anzuzeigen, verwenden Sie \verb|\[ formel \]|. Die Formel wird in einer 
eigenen Zeile angezeigt, aber nicht nummeriert. Also zum Beispiel \[ \cancelto{bla}{\sum_i^5 \sqrt[3]{\pi^2}} \] ohne Nummer.\\
\ \\
Besonders wichtige Formeln sollten Sie nummerieren und mit einem label versehen. So können Sie sich darauf beziehen. Dazu 
müssen Sie die equation-Umgebung nutzen. Beispiel:

\begin{equation}
\left( \int_{\underline{\theta}_b}^{\overline{\theta}_b} Q_b(s) ds\right) \underbrace{F_b}_{= 1} (\overline{\theta}_b) - \left( \int_{\underline{\theta}_b}^{\underline{\theta}_b} Q_b(s) ds\right) \underbrace{F_b(\theta_b}_{0} - \int_{\underline{\theta}_b}^{\overline{\theta}_b} F_b(\theta_b) Q_b(\theta_b) d\theta_b
\label{meineFormel}
\end{equation}
