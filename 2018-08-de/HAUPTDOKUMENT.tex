% Jedes Dokument braucht eine Documentclass
% Diese beeinflusst (auch) welche befehle es gibt: Ein \chapter
% gibt es zum Beispiel in book, aber nicht in article.
\documentclass{article}

% Präambel
% ab hier bis \begin{document}

\usepackage[utf8]{inputenc}
\usepackage[T1]{fontenc}

\usepackage[a4paper]{geometry}
% ermöglicht detailierte Einstellungen aller Ränder und Abstände
% meistens macht eine documentclass sinnvolle Vorgaben

\usepackage{graphicx}

\usepackage[french,english,ngerman]{babel}
% babel hat Auswirkungen auf Bezeichnungen im Dokument
% die zuletzt aufgelistete Sprache ist nun Standard für das Dokument
% statt "Contents" schreibt LaTeX nun "Inhaltsverzeichnis"


\usepackage{amsmath} % unbedingt für Mathe
\usepackage{amssymb} % unbedingt für Mathe
\usepackage{mathtools} % erweitert die beiden vorherigen
\usepackage{cancel} % erlaubt das Durchstreichen von Formel-Bestandteilen

\usepackage[marginal]{footmisc}
% beeinflusst die Darstellung von Fußnoten
% 'marginal' bewirkt, dass bei mehrzeiligen Fußnoten alle exakt an der gleichen Stelle anfangen

\usepackage[hidelinks]{hyperref}
% verlinkt unter anderem das Inhaltsverzeichnis
% die Option 'hidelinks' deaktiviert die optische Hervorhebung
% wenn möglich als letztes Package in der Präambel laden

\title{Wir lernen \LaTeX}
\author{Rüdiger Voigt}

\begin{document}

\maketitle

\tableofcontents
\listoffigures

% mit \input fügen wir den Inhalt einer Datei einfach so
% hier ein. ideal um große Dokumente in logische Abschnitte 
% aufzuteilen.
\section{Einleitung}

\subsection{Mit Referenzen auf Labels Bezug nehmen}

Beachten Sie bitte Formel \eqref{meineFormel} auf Seite \pageref{meineFormel} im Abschnitt \ref{sec:mathe}


\subsection[Kurztitel für das Inhaltsverzeichnis]{Unterabschnitt mit einem viel zu langen Titel, weil ich mich nicht kurzen fassen kann und will}





\subsection{Aufzählungen}

\begin{itemize}
\item abc
\item def
\end{itemize}

\begin{enumerate}
\item eins
\item zwei
\end{enumerate}

\chapter{Mathe}

Es gibt drei Anzeigemodi:\\
\ \\
Beim ersten schreiben Sie Ihre Formeln, wie etwa $\sum_{0}^{\infty}$ mitten in den Fließtext, indem Sie diese zwischen zwei \$-Zeichen aufschreiben. Die Darstellung ist platzsparend, aber bei komplexen Formeln schwer zu lesen. Falls Sie das Dollarzeichen im text brauchen, müssen Sie es mit einem Backslash escapen, d.h. \verb|100\$| ergibt 100\$.\\
\ \\
Daneben gibt es den abgesetzen Modus, der die Formel in eine eigene Zeile setzt. Hier ist die Darstellung deutlicher:\[ \sum_{0}^{\infty} \]
\ \\
Die dritte Methode ist die equation-Umgebung. Hier wird die Formel abgesetzt dargestellt und zusätzlich nummeriert.
\begin{equation}\label{einstein}
\sum_{0}^{\infty}
\end{equation}
Wenn Sie der equation ein Label geben (nach dem Beginn der Umgebung!), können Sie mit \verb|\ref{nameDesLabel}| oder \verb|\eqref{nameDesLabel}| überall im Dokument darauf Bezug nehmen. \verb|\ref| wird dabei ohne Klammern dargestellt: \ref{einstein}. Der Befehl \verb|\eqref| setzt automatisch Klammern: \eqref{einstein}.
\section{Floats}

% die Figure wird nicht einfach an Ort und Stelle eingefügt, sondern
% da wo es Sinn macht. Ich übergebe mit [htpb] Präferenzen:
% h = here
% t = top (oben auf einer Seite)
% p = page (gesammelt mit anderen Figures auf einer Seite)
% b = bottom (unten auf einer Seite)
\begin{figure}[htpb]
\begin{center}
\includegraphics[width=0.5\textwidth]{../img/Dala}
% width auf 50 prozent der Breite des Textkörpers gesetzt
% mit .. eine Ebene hoch
\end{center}
\caption[Katze]{Bildunterschrift}
\label{img:katze} % das Label NACH dem Element setzen damit es sich auf das richtige bezieht!
\end{figure}


\end{document}
