\documentclass[]{article}
\usepackage{lmodern}
\usepackage{amssymb,amsmath}
\usepackage{ifxetex,ifluatex}
\usepackage{fixltx2e} % provides \textsubscript
\ifnum 0\ifxetex 1\fi\ifluatex 1\fi=0 % if pdftex
  \usepackage[T1]{fontenc}
  \usepackage[utf8]{inputenc}
\else % if luatex or xelatex
  \ifxetex
    \usepackage{mathspec}
  \else
    \usepackage{fontspec}
  \fi
  \defaultfontfeatures{Ligatures=TeX,Scale=MatchLowercase}
\fi
% use upquote if available, for straight quotes in verbatim environments
\IfFileExists{upquote.sty}{\usepackage{upquote}}{}
% use microtype if available
\IfFileExists{microtype.sty}{%
\usepackage[]{microtype}
\UseMicrotypeSet[protrusion]{basicmath} % disable protrusion for tt fonts
}{}
\PassOptionsToPackage{hyphens}{url} % url is loaded by hyperref
\usepackage[unicode=true]{hyperref}
\hypersetup{
            pdftitle={Markdown Beispiel-Dokument},
            pdfauthor={Rüdiger Voigt, M.A.},
            pdfborder={0 0 0},
            breaklinks=true}
\urlstyle{same}  % don't use monospace font for urls
\usepackage{longtable,booktabs}
% Fix footnotes in tables (requires footnote package)
\IfFileExists{footnote.sty}{\usepackage{footnote}\makesavenoteenv{long table}}{}
\IfFileExists{parskip.sty}{%
\usepackage{parskip}
}{% else
\setlength{\parindent}{0pt}
\setlength{\parskip}{6pt plus 2pt minus 1pt}
}
\setlength{\emergencystretch}{3em}  % prevent overfull lines
\providecommand{\tightlist}{%
  \setlength{\itemsep}{0pt}\setlength{\parskip}{0pt}}
\setcounter{secnumdepth}{0}
% Redefines (sub)paragraphs to behave more like sections
\ifx\paragraph\undefined\else
\let\oldparagraph\paragraph
\renewcommand{\paragraph}[1]{\oldparagraph{#1}\mbox{}}
\fi
\ifx\subparagraph\undefined\else
\let\oldsubparagraph\subparagraph
\renewcommand{\subparagraph}[1]{\oldsubparagraph{#1}\mbox{}}
\fi

% set default figure placement to htbp
\makeatletter
\def\fps@figure{htbp}
\makeatother


\title{Markdown Beispiel-Dokument}
\author{Rüdiger Voigt, M.A.}
\date{}

\begin{document}
\maketitle

{
\setcounter{tocdepth}{3}
\tableofcontents
}
\section{Vorbemerkung}\label{vorbemerkung}

Es gibt mehrere Dialekte von markdown. Hier verwende ich die Variante,
welche pandoc versteht!

\newpage

\section{Beispiele Text}\label{beispiele-text}

\subsection{Fußnoten}\label{fuuxdfnoten}

Das ist ein Satz mit Fußnote\footnote{Inhalt der Fußnote}

\subsection{Aufzählungen}\label{aufzuxe4hlungen}

\begin{itemize}
\tightlist
\item
  a
\item
  b
\item
  c

  \begin{itemize}
  \tightlist
  \item
    zweite Ebene
  \end{itemize}
\item
  d
\end{itemize}

\section{Beispiele Mathe}\label{beispiele-mathe}

\(\underbrace{\hat{\mu} \sum_{0}^{\infty} \frac{x}{y}}_{\frac{\int_1^{\infty}\omega}{23}}\)

\section{Tabellen}\label{tabellen}

\begin{longtable}[]{@{}lll@{}}
\toprule
Warengruppe & Auf Lager & Bedarf\tabularnewline
\midrule
\endhead
Pinguine & 2 & 23\tabularnewline
Dropbears & 0 & 1000\tabularnewline
Dackel & 1 & 100000\tabularnewline
\bottomrule
\end{longtable}

\end{document}
