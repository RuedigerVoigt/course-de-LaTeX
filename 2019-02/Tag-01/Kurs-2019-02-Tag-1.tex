\documentclass[oneside, 10pt]{book}
% oneside ist für den Entwurf praktisch. Es schaltet ab, dass 
% Kapitel immer auf der rechten Seite beginnen müssen 
% (und Ähnliches).
%
% Sie setzen die Basisschriftgröße für eine normale Zeile. 
% LaTeX passt den Rest (Überschriften, etc.) daran an.
%
% Die Option draft ersetzt Bilder durch Rahmen 
% und markiert übervolle Zeilen.

\usepackage[utf8]{inputenc} % so ist die Datei encodiert
\usepackage[T1]{fontenc} % verhindert, dass Umlaute bei Copy-n-Paste zu zwei Buchstaben werden

\usepackage[a4paper]{geometry} 
% geometry für Seitenränder, Abstände aller Art, Papierformat, ...

% MATHE
\usepackage{amsmath}
\usepackage{amssymb}


% Blindtext erzeugen um floats et cetera vorführen zu können.
\usepackage{blindtext}


\usepackage[french,english,ngerman]{babel}
% Die zuletzt geladene Sprache wird Default für das Dokument.
% Hier also ngerman (Deutsch neuer Rechtschreibung).
% Sprachwechsel im Text mit \foreignlanguage{}{}.

\usepackage{graphicx} % Bilder einbinden

\usepackage{float}
 % Option H: erzwingt hier!

\usepackage{booktabs} % schöne Tabellen

\usepackage{lmodern}
% Hier setzen wir lmodern als Schrift für das Dokument.

\usepackage{hyperref}
% Verlinkung im Dokument
% Weblinks

\hyphenation{Binnen-schiff-fahrts-kapitäns-patent Boden-schleif-maschinen-verleih}
% Sie können Worttrennungen für das ganze Dokument hier hinterlegen.
% Bei mehrsprachigen Dokumenten beachten Sie die Antwort von egreg hier:
% https://tex.stackexchange.com/questions/182569/how-to-manually-set-where-a-word-is-split


\begin{document}

% automatisch generierte Verzeichnisse:
\tableofcontents % Inhaltsverzeichnis: erfasst Kapitel, Abschnitte, ...
\listoffigures % erfasst alle Bilder, die als figure eingebunden sind
\listoftables % erfasst alle Tabellen, die mit der table Umgebung eingebunden sind.


% Jedes Kapitel in eine eigene Datei.
% Die Dateien brauchen keine Präambel / müssen keine Packages laden.
% Übersetzt nach PDF wird aus dieser Hauptdatei.
% => plus an Übersicht
% => erleichtert die Fehlersuche

 \chapter{Einleitung}\label{chapterIntroduction}

\section{Querverweise}

Wenn Elemente ein Label haben, können Sie mit \verb|\ref| auf deren Nummerierung und mit \verb|\pageref| auf die Seite verweisen. Also zum Beispiel: Siehe Kapitel \ref{chapterIntroduction} auf Seite \pageref{chapterIntroduction}.\\
\ \\

\section{Sprache}

Um in anderen Sprachen als Englisch zu arbeiten, benötigen Sie das babel-package. Dann können Sie mit \verb|\foreignlanguage{}{}| mitten im Satz die Sprache wechseln:  \foreignlanguage{english}{\LaTeX\ is great!}

\section{Tabellen}

Die Float-Umgebung für Tabellen heißt table. Die eigentliche Tabelle ist eine tabular-Umgebung.

\begin{table}
\begin{center}
\begin{tabular}{rlc} % definiert die Spalten und deren Ausrichtung
% r = rechtsbündig
% l = linksbündig
% c = zentriert
\toprule % kommt aus booktabs
Variable & Wert & Beispiel\\
\midrule % braucht booktabs
a & 12345 6 & 3 \\
bbbbbbbbbbbbb & sedrftzuiopü\\
c & ertzuiop\\
\bottomrule
\end{tabular}
\label{tabMeine}
\caption{meine erste Tabelle}
\end{center}
\end{table}


\section{Neues aus der Wissenschaft}

\subsection{Subsection mit Nummerierung im Inhaltsverzeichnis}

\subsection*{Subsection, aber nicht im Inhaltsverzeichnis}
% Das Sternchen bewirkt, dass diese Subsection nicht im Inhaltsverzeichnis auftaucht.
% Sie wird auch nicht nummeriert, denn sonst gäbe es einen Sprung in der Nummerierung im 
% Inhaltsverzeichnis.
% Funktioniert so auch mit Kapiteln etc..

\subsection{Zeilenumbrüche}

Ein einfache Leerzeile

bewirkt, dass die erste Zeile des Folgeabsatzes eingerückt wird. Die zweite Zeile jedoch geht mit normalem linken Rand weiter und jede Folgende ebenso.

Das können Sie mittels Umbrüchen umgehen\footnote{Auch wenn das keine besonders schöbne Methode ist!}. \\
\ \\ % Achtung: hier muss ein erzwungenes Leerzeichen (Backslash + Leerzeichen) rein, sonst klappt der Umbruch nicht!
Das ist ein neuer Satz. 
\chapter{Mathe}

Es gibt drei Anzeigemodi:\\
\ \\
Beim ersten schreiben Sie Ihre Formeln, wie etwa $\sum_{0}^{\infty}$ mitten in den Fließtext, indem Sie diese zwischen zwei \$-Zeichen aufschreiben. Die Darstellung ist platzsparend, aber bei komplexen Formeln schwer zu lesen. Falls Sie das Dollarzeichen im text brauchen, müssen Sie es mit einem Backslash escapen, d.h. \verb|100\$| ergibt 100\$.\\
\ \\
Daneben gibt es den abgesetzen Modus, der die Formel in eine eigene Zeile setzt. Hier ist die Darstellung deutlicher:\[ \sum_{0}^{\infty} \]
\ \\
Die dritte Methode ist die equation-Umgebung. Hier wird die Formel abgesetzt dargestellt und zusätzlich nummeriert.
\begin{equation}\label{einstein}
\sum_{0}^{\infty}
\end{equation}
Wenn Sie der equation ein Label geben (nach dem Beginn der Umgebung!), können Sie mit \verb|\ref{nameDesLabel}| oder \verb|\eqref{nameDesLabel}| überall im Dokument darauf Bezug nehmen. \verb|\ref| wird dabei ohne Klammern dargestellt: \ref{einstein}. Der Befehl \verb|\eqref| setzt automatisch Klammern: \eqref{einstein}.
\chapter{Fazit}

\section{Beispiel für floats}

Zum Thema Platzierung von floats, sehen Sie sich bitte diese Seite an:\\
\url{https://tex.stackexchange.com/questions/39017}\\
\ \\
Hier erhalten beide floats die Platzierungsanweisung p und werden erwartungsgemäß beide auf eine gemeinsame Seite verschoben.

\Blindtext

% Ein Bild als figure:
% * es taucht im Bilderverzeichnis auf.
% * Sie können eine Bildunterschrift / caption vergeben.
% * Ein Kurztitel ist möglich.
% Sie können Präferenzen vergeben, wo das Bild angezeigt werden soll.
% Die Reihenfolge hat kein Gewicht, aber das Fehlen einer Option:
% h = here
% t = top of page
% b = bottom of page
% p = page for images
% H = (wenn float geladen ist) Anzeige hier erzwingen
\begin{figure}[p]
\begin{center}
\includegraphics[scale=0.5]{img/Dala}
\label{figDala}
\caption[Kurztitel Katze]{Das ist die Katze Dala im Jahr 2018.}
\end{center}
\end{figure}

\Blindtext

\begin{figure}[p]
\begin{center}
\includegraphics[width=0.4\textwidth]{img/toad}
\label{kroete}
\caption[Kurztitel Kröte]{Kröte, die sich in meinen Garten verirrt hat und zum nächsten Teich gebracht wurde.}
\end{center}
\end{figure}

\Blindtext




\section{Aufzählungen}

Aufzählungen sind eigene Umgebungen und lassen sich schachteln:

\begin{itemize}
\item abc
\item def  \begin{enumerate}
\item eins
\item zwei
\end{enumerate}
\end{itemize}






\url{http://www.uni-koeln.de}



\end{document}