 \chapter{Einleitung}\label{chapterIntroduction}

\section{Querverweise}

Wenn Elemente ein Label haben, können Sie mit \verb|\ref| auf deren Nummerierung und mit \verb|\pageref| auf die Seite verweisen. Also zum Beispiel: Siehe Kapitel \ref{chapterIntroduction} auf Seite \pageref{chapterIntroduction}.\\
\ \\

\section{Sprache}

Um in anderen Sprachen als Englisch zu arbeiten, benötigen Sie das babel-package. Dann können Sie mit \verb|\foreignlanguage{}{}| mitten im Satz die Sprache wechseln:  \foreignlanguage{english}{\LaTeX\ is great!}

\section{Tabellen}

Die Float-Umgebung für Tabellen heißt table. Die eigentliche Tabelle ist eine tabular-Umgebung.

\begin{table}
\begin{center}
\begin{tabular}{rlc} % definiert die Spalten und deren Ausrichtung
% r = rechtsbündig
% l = linksbündig
% c = zentriert
\toprule % kommt aus booktabs
Variable & Wert & Beispiel\\
\midrule % braucht booktabs
a & 12345 6 & 3 \\
bbbbbbbbbbbbb & sedrftzuiopü\\
c & ertzuiop\\
\bottomrule
\end{tabular}
\label{tabMeine}
\caption{meine erste Tabelle}
\end{center}
\end{table}


\section{Neues aus der Wissenschaft}

\subsection{Subsection mit Nummerierung im Inhaltsverzeichnis}

\subsection*{Subsection, aber nicht im Inhaltsverzeichnis}
% Das Sternchen bewirkt, dass diese Subsection nicht im Inhaltsverzeichnis auftaucht.
% Sie wird auch nicht nummeriert, denn sonst gäbe es einen Sprung in der Nummerierung im 
% Inhaltsverzeichnis.
% Funktioniert so auch mit Kapiteln etc..

\subsection{Zeilenumbrüche}

Ein einfache Leerzeile

bewirkt, dass die erste Zeile des Folgeabsatzes eingerückt wird. Die zweite Zeile jedoch geht mit normalem linken Rand weiter und jede Folgende ebenso.

Das können Sie mittels Umbrüchen umgehen\footnote{Auch wenn das keine besonders schöbne Methode ist!}. \\
\ \\ % Achtung: hier muss ein erzwungenes Leerzeichen (Backslash + Leerzeichen) rein, sonst klappt der Umbruch nicht!
Das ist ein neuer Satz. 