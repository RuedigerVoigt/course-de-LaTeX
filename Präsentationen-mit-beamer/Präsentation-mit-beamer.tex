\documentclass{beamer}


\usepackage[utf8]{inputenc}
\usepackage[T1]{fontenc}
\usepackage[french, english, ngerman]{babel}

% Probieren Sie verschiedene Themes einfach aus!

%\usetheme{Montpellier}
%\usetheme{Marburg}
\usetheme{Copenhagen}

\usepackage{amsmath}
\usepackage{amssymb}

% hyperref wird automatisch geladen!
%\usepackage{hyperref}

\usepackage{lmodern}


\title{Präsentationen mit \LaTeX}
\subtitle{-- die beamer-documentclass --}
\author{Rüdiger Voigt, M.A.}
\date{\today}


\begin{document}


\begin{frame}
\titlepage
\end{frame}

\begin{frame}
\frametitle{Übersicht}
\tableofcontents
\end{frame}

\section{Grundlagen}

\begin{frame}
\frametitle{Präsentationen mit \LaTeX}
Es ist ohne Weiteres möglich Präsentationen mit \LaTeX\ zu erstellen. Dabei eignet es sich ganz besonders für Vorträge zu wissenschaftlichen Themen.\\
\ \\
Was Sie in \LaTeX\ nur schwer umsetzen können sind animierte Übergänge oder Soundeffekte.
\end{frame}

\begin{frame}
\frametitle{Vorteile von \LaTeX\ gegenüber Powerpoint}
\begin{itemize}
\item Präsentationen werden ähnlich wie andere \LaTeX-Dokumente erstellt.
\item Ihnen stehen fast alle Funktionen von \LaTeX\ zur Verfügung. Dazu zählen unter anderem die mathematischen Formeln.
\item Sie können Inhalte leicht aus \LaTeX-Dokumenten in eine Präsentation kopieren.
\item Zur Anzeige Ihrer Präsentation benötigen Sie nur einen PDF-Viewer mit Vollbildfunktion -- zum Beispiel den Adobe Reader\footnote{STRG + L für Vollbild}.
\end{itemize}
\end{frame}

\begin{frame}
\frametitle{Warum beamer?}
Es gibt mehrere packages mit welchen Sie Präsentationen in \LaTeX\ erstellen können. Für beamer spricht:
\begin{enumerate}
\item eine hervorragende Dokumentation
\item einfache Befehle
\item gute und vielfältige Templates
\item schnelle Ergebnisse
\end{enumerate}
\end{frame}

\section{Minimalbeispiel}

\begin{frame}
\frametitle{Vorgehen}
\begin{itemize}
\item Legen Sie ein Dokument mit der documentclass beamer an.
\item Wählen Sie ein Theme und eventuell noch ein Farbtheme.
\item Speichern Sie in der Präambel die Basisinformation (title, author, \dots) und erstellen Sie eine titlepage.
\item Teilen Sie die Präsentation in sections auf.
\item Erstellen Sie die einzelnen Frames.
\item Kompilieren Sie die Version für den Vortrag und eventuell eine Version als Handout.
\end{itemize}
\end{frame}

\begin{frame}[fragile]
\frametitle{Minimalbeispiel}
\begin{verbatim}
\documentclass{beamer}
\usepackage[utf8]{inputenc}
\usepackage[T1]{fontenc}
\usepackage[english, ngerman]{babel}
\usetheme{Frankfurt}
\title{Präsentationen mit \LaTeX}
\subtitle{-- die beamer-documentclass --}
\author{Rüdiger Voigt, M.A.}
\date{\today}
\begin{document}
...
\end{document}
\end{verbatim}
\end{frame}

\begin{frame}[fragile]
\frametitle{Die Title-Page}
\verb|\begin{frame}|\\
\verb|\titlepage|\\
\verb|\end{frame}|\\
\ \\
Erzeugt eine Titelseite mit den Informationen aus der Präambel (Titel, Untertitel, Autor, Datum, ...)
\end{frame}

\begin{frame}[fragile]
\frametitle{Eine einzelne Folie}
\verb|\begin{frame}|\\
\verb|\frametitle{Kopfzeile}|\\
\verb|Inhalt der Folie ...|\\
\verb|\end{frame}|\\
\ \\
Falls Sie auf der Seite Code mit der verbatim-Umgebung ausgeben, dann müssen Sie der frame die Option fragile geben!
\end{frame}

\begin{frame}[fragile]
\frametitle{Strukturierung mit sections}
Sie sollten Ihre Präsentation mit \verb|\section{Abschnittstitel}| in Abschnitte aufteilen, denn das erzeugt bei den meisten Themes eine übersichtliche Navigationsstruktur, die ihren Zuhörern anzeigt wo Sie im Vortrag sind.\\
\ \\
Beamer lädt automatisch hyperref und die Navigationselemente werden automatisch verlinkt. Damit können Sie schnell zu einem bestimmten Thema springen.
\end{frame}

\begin{frame}[fragile]
\frametitle{Weitere Features}
\begin{itemize}
\item Sie können unter anderem mit dem \verb|\pause| Befehl Inhalte schrittweise einblenden. Jeder Schritt entspricht dabei einer neuen Seite im PDF. Für Handouts können Sie automatisiert eine Version erstellen, die das auf eine Einzelseite reduziert.\pause
\item Es ist möglich Informationen für die vortragende Person auf einem zweiten Bildschirm auszugeben.
\end{itemize}
\end{frame}

\begin{frame}
\frametitle{Feature: Blöcke}
\begin{theorem}[Titel für das Theorem]
Theorem
\end{theorem}

\begin{alertblock}{ein Alertblock}
für besonders wichtige Informationen
\end{alertblock}
\end{frame}


\section{Anleitungen}

\begin{frame}
\frametitle{Offizielle Anleitung}
Die offizielle Anleitung hat ein Kapitel zum Einstieg, dass Ihnen alles Wesentliche anschaulich vermittelt. Der Rest der Anleitung geht ins Detail.\\
\ \\
Sie finden Sie hier: \url{https://ctan.org/pkg/beamer}
\end{frame}



\begin{frame}
\frametitle{Informationen zum Kurs}
Die Unterlagen zum Kurs finden Sie hier:\\
\url{https://www.ruediger-voigt.eu/latex-kurs.html}\\
\ \\
Die .tex-Datei dieser Präsentation und weitere Beispiele finden Sie im github-Repository des Kurses:\\
\url{https://github.com/RuedigerVoigt/course-de-LaTeX}
\end{frame}
\end{document}